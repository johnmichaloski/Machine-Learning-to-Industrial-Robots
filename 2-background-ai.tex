\section{Background AI}
Artificial intelligence is the study of ideas that enable computers to be intelligent~\cite{winston1992artificial}.  Since the formal definition of intelligence is difficult but the concept is intuitive, emphasizing the role of intelligent computing suffices to describe the overreaching goal of AI. AI itself has many applications. Figure~\ref{fg:AICapabilities} sketches some of the major capabilities under the umbrella of artificial intelligence.

AI is used for reasoning about problems, such as manipulating blocks in a blocks world that at a more practical level  corresponds to robot grasping and part assembly. AI is also used for exploring alternative routes to a problem solution using searching and backtracking. Searching techniques can be exhaustive, evolutionary, smart, and/or pruned. Such functionality is useful for planning and in games.  Problem solving is another key AI area where a generate-and-test or a rule-based-systems can be applied to computer architecture layout, health diagnoses, and game playing  Such system are known as ``expert systems'' that uses rules to make deductions or choices. Knowledge representation is another domain under the purview of AI often implemented as semantic nets to model natural language as well as human understanding.

Learning is a branch of artificial intelligence based on the idea that systems exhibit adaptive behavior and can learn from data, identify patterns and make decisions. Machine learning relies on a method of data analysis that facilitates model building.

Artificial intelligence as applied to robotics  includes sensing, navigation, path planning, and control as these areas require programming and by definition are artificial, while at the same time are intrinsically linked to a smarter more intelligent robot.


Robots typically include perception so that robot control for reaching and moving objects in a complex environment with obstacles and vision occlusions is a leading AI application~\cite{staley2018drl}.


\begin{figure}[!h]
\centering
%\framebox{
\includegraphics*[width=0.9\columnwidth]{./Figure/AICapabilities.jpg}
%}
\caption{General Capabilities of AI}
\label{fg:AICapabilities} 
\end{figure}


\subsection{Machine learning}
\label{chapter2}
\textit{}
%---- the scope of this chapter}}
Webster dictionary defines learning as the ability to gain knowledge or understanding of or skill in by study, instruction, or experience; or as the modification of a behavioral tendency by experience (such as exposure to conditioning)~\cite{learningdef}. Of interest in this definition, is the ability to gain an understanding through experience. For our purposes, we refine the term experience to be data training. This leads to the following machine learning definition. 

Artificial Intelligence is a more general concept of ``smarter'' machines while machine learning is an AI application that ``smart'' machines continually adapt to data themselves. The most frequently cited definition of learning comes from Simon~\cite{simon1983should}: ``Learning denotes changes in the system that is adaptive in the sense that they enable the system to do the same task or tasks drawn from the same population more effectively the next time''.  But there can be a more static viewpoint of machine learning. One can use machine learning and big data techniques to train neural net to be intelligent and make  intelligent predictions, without continuously adapting to the environment and improving.  However, this does assume that the trained model is sufficiently intelligent for the  application.

Variations on machine learning terminology and implementation exist. Some of the terminology and implementations overlap. The following sketches some of the most salient learning concepts.

\begin{description}

%----  Deep learning
\item{\textbf{Deep learning}}
Deep learning is the science of training large artificial neural networks~\cite{pierson2017deep}. Deep neural networks (DNNs) form compact
representations from raw, high-dimensional, multimodal sensor data commonly
found in robotic systems~\cite{bohmer2015autonomous}.


%----------------------------------------------------------------------------------------


\item{\textbf{Supervised Learning}}
Type of learning in which the data outcome is known, and this data outcome is explicitly used during training that is the model is trained under the supervision of a teacher. For example, if we want to build a classification model for handwritten digits, the input will be the set of images (training data) and the target variable will be the labels assigned to these images, that is their digit value outcomes from 0-9 associated with each image.

\item{\textbf{Unsupervised Learning}}
Type of learning in which no supervisor is involved.   The goal is to directly infer the properties of a dataset without the help of a supervisor providing correct answers for each input.

\item{\textbf{Imitation learning}}
Type of learning in which the goal is to train by replicating the observed behavior and is typically achieved using supervised learning techniques.

%----  Robot Reinforcement learning
\item{\textbf{Reinforcement learning}}
Reinforcement learning is a type of learning algorithm in which the machine takes decisions on what actions to take, given a certain situation/environment, so as to maximize a reward. In reinforcement learning, learner is not told which actions to take, as in most forms of machine learning, but instead, must discover which actions yield the most reward by trying them. In the most interesting and challenging cases, actions may affect not only the immediate reward but also subsequent future  rewards.\\
\\

\end{description}


%----------------------------------------------------------------------------------------
\section{Robotic AI Applications}
The goal of AI is to develop worthwhile robotic applications that traditional programming approaches have had difficulty. Typically, a robot software system is divided into planning, sensing and control. Robot applications typically mimic humans, such as pick and place requiring dexterous grasping, planning, visual sensing and tracking, obstacle avoidance.   This section will review some of the current robotic AI/ML research.

%----  Robot Reinforcement learning
\subsection{Reinforcement Learning }
Reinforcement Learning (RL) is a machine learning framework for optimizing the behaviour of an agent interacting with an unknown environment~\cite{sutton1998introduction}.
Reinforcement Learning enables a robot to autonomously discover an optimal behavior through trial-and-error interactions
with its environment~\cite{kormushev2013reinforcement}. Instead of explicitly detailing the solution to a problem, in reinforcement learning the designer of a control task provides feedback in terms of an objective function that measures the one-step performance of the robot.

Reinforcement learning enables a robot to autonomously discover an optimal behavior through trial-and-error interactions
with its environment\cite{kormushev2013reinforcement}. Instead of explicitly detailing the solution to a problem, in reinforcement learning the designer of a control task provides feedback in terms of an objective function that measures the one-step performance of the robot.



Larouche and F\'{e}raud~\cite{laroche2017reinforcement} defines reinforcement Learning (RL) to be considered learning through trial and error to control an agent behavior in a stochastic environment: at each time step $t \in N$, the agent performs an action $a(t) \in  A $, and then perceives from its environment a signal $o(t) \in \omega$ called observation, and receives a reward $t(t) \in R$,  bounded between $R_{min}$ and $R_{max}$. Laroche and F\'{e}raud then proposes to share their trajectories expressed in a universal format. A high level definition of the RL algorithms allows to share trajectories between algorithms: a trajectory as a sequence of observations, actions, and rewards can be interpreted by any algorithm in its own decision process and state representation. 

Kober~\cite{kober2013reinforcement} uses training a robot to play table tennis to explain RL concepts. Robot observations of ball position and velocity as well as the internal joint dynamics constitute the \textit{state} $s$ of the system. The \textit{actions} $a$ available to the robot could be torque motor commands. A function $\pi$ generates the actions based on the state and would be called a \textit{policy}. This leads to the definition of a reinforcement problem is to find a policy that optimizes the long-term sum of \textit{reward} $R(s,a)$.



%----  Robot Grasping learning
\subsection{Grasping}
The broader goal of robot grasping is to develop highly reliable robot grasping across a wide variety of rigid objects such as tools, household items, packaged goods, and industrial parts.

The Dexterity Network (Dex-Net) is a research project at the University of California Berkeley Automation Lab whose goal is to generate datasets of synthetic point clouds, robot parallel-jaw grasps and metrics of grasp robustness based on physics for thousands of 3D object models to train machine learning-based methods to plan robot grasps~\cite{mahler2017learning,dexnet}. Several generation of Dex-Net have evolved adding various grasping skills. Of distinction, Dex-Net 2.0 is designed to generated training datasets to learn grasping models that predict the probability of success of candidate parallel-jaw grasps on objects from point clouds. The goal was for a robot to quickly plan grasps for a wide variety of objects. 

\subsection{Robot Vision and Sensory Processing}
Machine learning with deep convolutional neural networks has become a primary method when it comes to image classification problems. Such  sensory AI technology need not be limited to image sensor processing, but could be applied to other sensor applications such as tactile sensing or joint encoders~\cite{bohmer2015autonomous}.



\subsection{Highly Redundant Robot Kinematic}
Highly redundant robots are prime candidates for machine learning to solve the inverse kinematics. Especially for redundant robots,  inverse kinematics algorithms need to address how to determine a particular solution in face of multiple solutions. 

In \cite{973374},   a statistical learning algorithm, ``Locally Weighted Projection Regression'', is shown to exhibit efficient learning of inverse kinematic mappings in an incremental fashion even when input spaces are high dimensional.




%%%%%%%%%%%%%%%%%%%%%%%%%%%%%%%%%%%%%%%%%%%%%%%%%%%%%%%
%% Keras overview for hands on 
%
\subsection{Machine Learning Technology Case Study}
% Keras documentation: https://keras.io/scikit-learn-api/
Fran{\c{c}}ois Chollet developed Keras~\cite{chollet2018keras} to simplify the creation of deep learning models. Keras is a good starting point to understand AI/ML technology as Keras points out, it is an API designed for human beings, not machines. Keras reduces ML programming complexity while providing a powerful porgramming tool~\cite{KerasBenefits}. 

Keras is an open source neural network library written in Python capable of running on top of several prominent neural net technologies: TensorFlow~\cite{abadi2016tensorflow}, Microsoft Cognitive Toolkit~\cite{CNTK}, or Theano~\cite{Theano}. Keras provides a straightforward deep neural network programming framework that is evolving into  a  a ``push-button'' automated machine learning where the daunting combinatorics of selecting parameters and configuration are deduced from experimental evaluation~\cite{jin2018efficient}.

For the present, the basic steps in developing a Keras deep machine learning application are:

\begin{enumerate}
\item Define the data with various tools to assist in creating, managing and filtering the data.
\item Define a neural network model - either sequential or Functional API
\item Configure and compile the learning process
\item Train (fit) the model
\item Save the model for reuse
\item Load the model either
\item Predict with the model

\end{enumerate}

\begin{description}
\item{ \textbf{Define the data}}\\
Keras preprocessing is the data preparation and data augmentation module of the Keras deep learning library. It provides utilities for working with image data, text data, and sequence data.

\textul{Sequence Processing}
\begin{description}

\item{\textbf{TimeseriesGenerator}} - Utility class for generating batches of temporal data.
\item{\textbf{pad\_sequences}} - Pads sequences to the same length.
\item{\textbf{skipgrams}} This function transforms a sequence of word indexes (list of integers) into tuples of words~\cite{mikolov2013efficient}.
\item{\textbf{make\_sampling\_table}} - Generates a word rank-based probabilistic sampling table.  Used for generating the sampling\_table argument for skipgrams. sampling\_table[i] is the probability of sampling the word i-th most common word in a dataset (more common words should be sampled less frequently, for balance).

\end{description}

\item {\textbf{Configure and compile the learning process}}\\
Before training a model, you need to configure the learning process, which is done via the compile method. It receives three arguments:

\begin{itemize}
\item An optimizer. This could be the string identifier of an existing optimizer (such as rmsprop or adagrad), or an instance of the Optimizer class. See: optimizers.
\item Loss function. This is the objective that the model will try to minimize. It can be the string identifier of an existing loss function (such as categorical\_crossentropy or mse), or it can be an objective function. 
\item List of metrics. For any classification problem you will want to set this to metrics=['accuracy']. A metric could be the string identifier of an existing metric or a custom metric function.
\end{itemize}

\end{description}
% from https://keras.io/getting-started/sequential-model-guide/#compilation 
\subsection{Specifying the input shape}
The model needs to know what input shape it should expect. For this reason, the first layer in a Sequential model (and only the first, because following layers can do automatic shape inference) needs to receive information about its input shape. There are several possible ways to do this:

Pass an input\_shape argument to the first layer. This is a shape tuple (a tuple of integers or None entries, where None indicates that any positive integer may be expected). In input\_shape, the batch dimension is not included.
Some 2D layers, such as Dense, support the specification of their input shape via the argument input\_dim, and some 3D temporal layers support the arguments input\_dim and input\_length.
If you ever need to specify a fixed batch size for your inputs (this is useful for stateful recurrent networks), you can pass a batch\_size argument to a layer. If you pass both batch\_size=32 and input\_shape=(6, 8) to a layer, it will then expect every batch of inputs to have the batch shape (32, 6, 8).

\subsectin{Compilation}
Before training a model, you need to configure the learning process, which is done via the Keras compile method. You configure a Keras model for training using the following command:
\begin{minted}{python}
compile(object, optimizer, loss, metrics = NULL, loss_weights = NULL,
  sample_weight_mode = NULL, weighted_metrics = NULL,
  target_tensors = NULL, ...)
\end{minted}
Figure~\ref{fg:keras_compilation} shows the Keras compilation argument tree. 

\begin{figure}[!h]
\centering
%\framebox{
\includegraphics*[]{./Figure/KerasCompileModel.jpg}
%}
\caption{Keras Configuration using Compile Command}
\label{fg:keras_compilation} 
\end{figure}

It receives three arguments:
\begin{description}


\item{\textbf{ optimizer}} This could be the string identifier of an existing optimizer (such as rmsprop or adagrad), or an instance of the Optimizer class. See: optimizers.
\item{\textbf{Loss function}} This is the objective that the model will try to minimize. It can be the string identifier of an existing loss function (such as categorical_crossentropy or mse), or it can be an objective function. See: losses.
\item{\textbf{list of metrics}} For any classification problem you will want to set this to metrics=['accuracy']. A metric could be the string identifier of an existing metric or a custom metric function.
\end{description}

\subsectin{Training}
Keras models are trained on Numpy arrays of input data and labels. For training a model, you will typically use the fit function. 





\subsectin{Initialization}

What is the purpose of setting an initial weight on deep learning model?

This is greatly addressed in the Stanford CS class CS231n:

Pitfall: all zero initialization. Lets start with what we should not do. Note that we do not know what the final value of every weight should be in the trained network, but with proper data normalization it is reasonable to assume that approximately half of the weights will be positive and half of them will be negative. A reasonable-sounding idea then might be to set all the initial weights to zero, which we expect to be the “best guess” in expectation. This turns out to be a mistake, because if every neuron in the network computes the same output, then they will also all compute the same gradients during backpropagation and undergo the exact same parameter updates. In other words, there is no source of asymmetry between neurons if their weights are initialized to be the same.

There are several weight initialization strategies; each one is best suited for a type of activation function. For instance, Glorot's initialization aims at not saturating sigmoid activations, while He's initialization is meant for Rectified Linear Units (ReLUs).

Loss Functions

\begin{tabular}{ l l l }
Loss Function &	ML Application &	Description \\
Cross-Entropy&	Classification,
Binary classification &	Cross-entropy loss, or log loss, measures the performance of a classification model whose output is a probability value between 0 and 1.\\
Hinge &	classification	& &\\
Huber &	regression		& &\\
Kullback-Leibler (KL)	& &	The KL divergence tells us how well the probability distribution Q approximates the probability distribution P by calculating the cross-entropy minus the entropy. \\
MAE (L1) &&\\		
MSE (L2)	&&\\		
\end{tabular}

